\documentclass{article}

\usepackage[utf8]{inputenc}
\usepackage{xeCJK}
\usepackage[main=english]{babel}

\usepackage[a4paper, margin=1.5in]{geometry}
\usepackage{indentfirst}
\usepackage[skip=11pt, indent=20pt]{parskip}
\def\thesection{\arabic{section}}


\title{The Bard and The Painter}
\author{Apollyn Rivers}
\date{August 2022}

\begin{document}

\maketitle

\begin{quote}
    For all the chestnuts, maple trees, lyres, and painters who wish to have a bard.
\end{quote}

\section{} %1
The Painter met The Bard in autumn, not in a loving spring, not in a playful summer, and not in a cuddly winter. The Painter met The Bard in the autumn when maple leaves hadn't fallen off the trees, and when the chestnuts hadn't gotten sweetened..

So if we said it's The Bard who made the maple leaves red, and made the chestnuts sweet for The Painter, it would not be a false statement, right?

\section{} %2
The first time when The Painter saw The Bard, he was singing next to the fountain.

The Bard seemed to come from somewhere which was far far away. He sang for a long time, and he sang a lot of different songs. Some of them The Painter had heard before, but there were more which he had not.
The Painter couldn't help stopping by. He had never left this country. Never had he seen the starry night in a desert, nor the aurora in northern land. Never had he met the beasts in the black forest, nor the krakens in the abyss. Never had he heard of the horns right before a battle, nor had he experienced the wildness in a carnival.

In the songs sung by The Bard, The Painter finally saw all the things he was searching for.

\section{} %3
The first time The Bard saw The Painter, The Painter was sketching from nature on the bridge.

The Bard had been to many places, and had seen paintings done by many people. There were both muddy doodles painted by innocent kids, and epic tales drawn with jewellery and blood by kings and politicians.

But The Bard still held his breath when he saw The Painter's work.

“I can see his soul there.” The Bard thought.

\section{} %4
The Bard went to see The Painter work on the bridge every single day.

Sometimes the kids who recognised him will beg him for a few stories. Sometimes he happily obliged, sometimes he didn't. Most of his stories were not suitable for kids, after a while the kids stopped coming.

This is perfectly fine. His stories and songs were for another person, a rather special person.

“Hey, you don't know me. But I am not a bad person. I like your paintings, it makes me wanna sing. Would you like to hear them?”

Would this make me sound like a pervert? The Bard was as stressed as always.

\section{} %5
The Bard appeared on the bridge every single day.

Sometimes the kids who recognised him will beg him for a few stories. Sometimes he happily obliged, sometimes he didn't. Maybe it's because most of his stories were not suitable for kids, after a while the kids stopped coming.

It was not a problem. The Painter was not a kid, he liked those stories.
But the children don't come anymore, The Bard doesn't tell stories anymore. What excuses do I have to ask him to sing me a song or tell me a story?

“Hey, you don't know me. But I'm not a bad person. I like your songs, they make me want to draw, would you like to see them?”

\textit{This sounds a bit like a love confession.}
The Painter was as shy as usual.

\section{} %6
The Painter's canva got blown away by the wind. All the accessories were scattered on the ground.

The Painter was sad. A lot of those wasted paints were newly bought. The Bard was also sad. He has been waiting for this drawing to be completed for one long week.

The Bard walked towards The Painter to help him pick up the stuff. When he straightened himself back up, he saw the surprised expression on The Painter's face. He felt a bit embarrassed all of sudden. He gave all the things to The Painter, turned around and tried to flee the scene.

\textit{I might have scared him.}

The Bard was a bit disappointed with himself.

\section{} %7
\textit{I might have scared him.}

Seeing how The Bard was trying to flee, The Painter finally realised it.
But what a good chance this was. Putting the shyness and embarrassment aside, The Painter grabbed The Bard's hand.

It was warm, and dry. And there were some minor calluses on it. They probably came from playing the strings for a long time.

Staring at The Bard's face, he could see bewilderment on it.  The Painter went blank all of a sudden not knowing what to do.

“I, um, well, could you— could you sing me a song?”

\textit{Oh mother in heaven, please take me with you.}

\section{} %8
The Bard blinked a few times. He looked at the hand that's still holding him tightly. He looked at The Painter who had desperation written all over himself in capitals. He blinked again. He's not too sure what was going on.

But that was okay. Go deep when it's simple; go straight when it's complicated. The Bard heard this from an old fisherman a long time ago. He found it very wise.

Life was simple, until we made it complicated by ourselves.

He wanted me to sing a song for him, so I would sing a song for him.

So The Bard did.

There were forests and seashores. There were swaying corn fields and monstrous scarecrows. Of course there were also elegant elves and grumpy dragons. In the song there was everything The Bard had seen in The Painter's drawings.

The Bard knew, The Painter would understand.

The Painter was speechless.

The Bard tilted his head to the side and gave The Painter puppy eyes and a huge smile.

“Now I have sung the song for you. Would you do a painting for me?”

\section{} %9
Of course The Painter did not give The Bard a painting. But he took The Bard home, and gave him a cup of hot chocolate.
\par The Painter asked The Bard why he wanted him to do a painting for him, and what kind of painting he wanted.
\par The Bard didn't answer the questions directly.
\par “Why did you ask me to sing me a song?” The Bard asked The Painter.
\par “Because I heard what I have wanted to draw in your songs.” No need to think twice, The Painter answered immediately.
\par And then he saw The Bard smiling at him. Almost like he was saying “See? Haven't you found the answer already?”

\section{} %10
The Painter and The Bard became friends.

The Painter would demonstrate the texture and material of each different kind of paper and paints to The Bard. The Bard would explain how flexible rhythm and rhymes could be.

Compared to the result, the process of creation was more interesting.

Whether they were completed or incomplete, The Bard liked each and every drawing from The Painter. In his eyes, all the strokes made beautiful phrases.

Whether they were fully phrased or scattered, The Painter liked each and every melody The Bard made. In his ears, all the paragraphs made beautiful scenes. 

\section{} %11
Finally, the chestnuts were sweet and the maple leafs were red.

The Painter and The Bard sat on the couch together eating chestnuts. There was a saucer filled with maple syrup on the side.

“I often think that art is similar to chestnuts.” The Painter said while opening a chestnut with full force: “Only when you get through the hard shell, you get to taste the sweet core.”

“I often think that poets are similar to maple trees.” The Bard licked the remaining maple syrup off the spoon: “Only after being hurt, they get to produce sweet content.”

They stared at each other.

“It's good you are not a maple tree. I don't like seeing you get hurt.” The Painter said.

It's good you are not a chestnut either. Men and chestnuts can't hug each other.” The Bard said.

The Painter continued with eating the chestnuts.

The Bard continued with licking off the maple syrup.

\textit{
Thankfully you are who you are and I am who I am. We are not chestnuts, nor are we maple trees.
We don't need to get hurt. We can hug each other.}

\section{} %12
\par Winter arrived soon.
\par The Bard checked out from the hotel and moved in with The Painter.
\par That was the warmest winter ever.

\section{} %13
\par The Bard fell in love with <The Nightingale and the Rose> recently.
\par It was almost like The Painter got a nightingale in his house.
\begin{quote}
“Give me a red rose. Give me a red rose. I am going to give it to that young man, in exchange for love. Give me a red rose. Give me a red rose. I will build it out of music by moonlight. I will stain it with my own heart's blood. With my breast against a thorn, I will sing. Until it pierces my heart, I will sing. All night long; until my blood and my song have made it the reddest rose you have ever seen, I will sing.
Give me a red rose, I want to give it to that young man.”
\end{quote}
\par The Bard sang it over and over again.
The Painter couldn't hold himself anymore: “Are you in love?”
The Bard smiled cheerfully: “I am always in love. I am in love with myself. I am in love with my music and my poems. I love life because only when it flourishes, poems can grow. You make my life colourful, so I love you.”
\par The Painter frowned: “I don't want to be that young man.” I won't let your rose be run over by the cart-wheel.
\par The smile on The Bard's face got wider: “You are not the young man, you are the tree.”
\par Without the tree who looked forward to the singing of the little nightingale, how would she be able to build the rose?

\section{} %14
\par The Painter has locked himself in his room for days.
\par His work got shredded by his master.
\par "You will never become an artist who is as good as The Court Painter!” His master shouted at him.
\par By the way, The Court Painter is the other apprentice of his master.
"Look at these lines, messy and useless! And these colour blocks, dirty and meaningless!”
\par "Why would I get such an apprentice like you?”
\par The Painter wanted to argue. It was not like that. Every stroke had its meaning. This one was for resistance. That one was about hope. And that one at the corner represented buried love. Each colour had its own words as well. Red spoke love. Green shouted freedom. Blue didn't speak, it was always quiet. And that block of warming brown was the chestnut that came from that autumn, which got sweetened because of that special someone.


\end{document}
